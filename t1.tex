\documentclass[a4paper,11pt,twoside,twocolumn]{article}
\usepackage{amsmath}
\usepackage{adjmulticol}
\usepackage{mathtools}
\usepackage{algpseudocode}
\usepackage{sectsty}
\usepackage{fancyhdr}
\pagestyle{fancyplain}
\fancyhead{}
\fancyfoot[L]{}
\fancyfoot[C]{}
\fancyfoot[R]{\thepage}
\renewcommand{\headrulewidth}{0pt}
\renewcommand{\headheight}{13.6pt}


\DeclarePairedDelimiter\ceil{\lceil}{\rceil}
\DeclarePairedDelimiter\floor{\lfloor}{\rfloor}

\newcommand{\horrule}[1]{\rule{\linewidth}{#1}}
\title{
  \normalfont \normalsize
  \horrule{0.5pt}
  \huge T1-S1
  \horrule{2pt}
}  
\author{Kaung Htet - A0105860L}
\date{20 August 2014}

\begin{document}


% (a)
\subsection{(a)}

\begin{center}
  \begin{tabular}{|c|c|c|}
    \hline
    n & $h(n)$ & $d(n)$ \\ \hline
    1 & 1 & 0 \\
    2 & 2 & 1 \\
    3 & 2 & 2 \\
    4 & 3 & 2 \\
    5 & 3 & 3 \\
    6 & 3 & 3 \\
    7 & 3 & 3 \\
    8 & 4 & 3 \\
    9 & 4 & 4 \\
    10 & 4 & 4 \\
    11 & 4 & 4 \\
    12 & 4 & 4 \\
    13 & 4 & 4 \\
    14 & 4 & 4 \\
    15 & 4 & 4 \\
    16 & 5 & 4 \\
    17 & 5 & 5 \\
    18 & 5 & 5 \\
    19 & 5 & 5 \\
    20 & 5 & 5 \\
    21 & 5 & 5 \\
    22 & 5 & 5 \\
    23 & 5 & 5 \\
    24 & 5 & 5 \\
    25 & 5 & 5 \\
    31 & 5 & 5 \\
    32 & 6 & 5 \\
    33 & 6 & 6 \\
    63 & 6 & 6 \\
    64 & 7 & 6 \\
    65 & 7 & 7 \\
    100 & 7 & 7 \\
    127 & 7 & 7 \\
    128 & 8 & 7 \\
    129 & 8 & 8 \\
    100 & 7 & 7 \\
    1023 & 10 & 10 \\
    1024 & 11 & 10 \\
    1025 & 11 & 11 \\
    $10^6$ & 20 & 20 \\
    $10^9$ & 30 & 30 \\ \hline

  \end{tabular}
\end{center}

\subsection{(b)}

\begin{multicols}{2}
  
  \begin{algorithm}
    \caption{Repeated-halving:$h(n)$}\label{Repeated-halving:$h(n)$}
    \begin{algorithmic}
      \State $step\gets 0$
      \While{$n\neq 0$}
        \State $n\gets n/2$
        \State $step\gets step+1$
      \EndWhile
      \State \textbf{return} $step$
    \end{algorithmic}
  \end{algorithm}
 
\columnbreak
 
  \begin{algorithm}
    \caption{Repeated-doubling:$d(n)$}\label{Repeated-doubling:$d(n)$}
    \begin{algorithmic}
      \State $start\gets 1$
      \State $step\gets 0$
      \While{$start < n$}
        \State $start\gets start*2$
        \State $step\gets step+1$
      \EndWhile
      \State \textbf{return} $step$
    \end{algorithmic}
  \end{algorithm}

\end{multicols}
\subsection{(c)}
In repeated-halving process, the number $n$ gets halved in each
iteration. But the iteration doesn't stop when $n$ reaches to 1. It
goes an additional iteration and stops when $n$ reaches to
0. The number of steps will be equivalent to the floor of the
$log_2$ plus one for the additional iteration.

However, in repeated-doubling process, the number $n$ gets doubled and
it starts from 1. So the number of steps is the ceiling of $log_2$.
\subsection{(d)}

\begin{equation*}
  h(n) = \floor*{\log_2 n} + 1
\end{equation*}
\begin{equation*}
  d(n) = \ceil*{\log_2 n}
\end{equation*}

\end{document}
